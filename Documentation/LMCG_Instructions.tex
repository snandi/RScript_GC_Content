\documentclass[11pt]{article}
%\usepackage{fancyhdr}
\usepackage[hmargin=3cm,vmargin=3cm]{geometry}
\usepackage{dsfont}
\usepackage{bbold}
\usepackage{graphicx, float}
\usepackage{verbatim}
\usepackage{amssymb, amsmath}
%\usepackage[hmargin=3cm,vmargin=3cm]{geometry}
\usepackage{wrapfig}
\usepackage{xcolor}
\DeclareGraphicsExtensions{.pdf,.png,.jpg, .jpeg}

\begin{document}

\title{Instructions for using LMCG code libraries to access necessary data}
\author{Subhrangshu Nandi\\
  Department of Statistics\\
  Laboratory of Molecular and Computational Genomics\\
  nandi@stat.wisc.edu}
%\date{July 13, 2013}
\maketitle

\begin{enumerate}
\item
In general, all perl libraries written by Steve, will be in the folder \begin{verbatim} /home/steveg/bin \end{verbatim}
\item
To extract data on nanocoding, particularly, for the GC-Intensity project, go to \begin{verbatim} /home/steveg/bin/nMaps/GC_content/ \end{verbatim} \\
Use the file \texttt{alignmentLocations.pl} to produce tables like \begin{verbatim} /aspen/steveg/human_nMaps/GC_content/alignmentChunks.withLength.all7134Groups.goldOnly \end{verbatim} which look like
\begin{verbatim}
refID refStartIndex refEndIndex opID opStartIndex opEndIndex refStartCoor 
refEndCoord opStartCoor opEndCoord orient lengthRatio

chr1 1 2 2388301_734_2060415 18 17 17621 19740 121536 119548 -1 1.08472434266327
chr1 1 2 2393020_734_2060368 15 14 17621 19740 113319 111465 -1 1.04471447859582
chr1 1 2 2396217_734_2060303 0 1 17621 19740 0 2060 1 0.95170660205211
chr1 1 2 2398509_734_2060043 0 1 17621 19740 0 2266 1 0.941316312241508
chr1 2 3 2387802_734_2061088 16 15 19740 28925 106350 99103 -1 1.0932487070992
chr1 2 3 2388301_734_2060415 17 16 19740 28925 119548 111623 -1 1.08472434266327
chr1 2 3 2388598_734_2060536 14 13 19740 28925 112074 102403 -1 1.03741278084123
chr1 2 3 2393020_734_2060368 14 13 19740 28925 111465 103128 -1 1.04471447859582
chr1 2 3 2396217_734_2060303 1 2 19740 28925 2060 11742 1 0.95170660205211
chr1 2 3 2398509_734_2060043 1 2 19740 28925 2266 11742 1 0.941316312241508
\end{verbatim}
\item
To list intensities at each pixel, for any fragment, use the file \texttt{fluorIntensityFor\_nMaps.byPixel.pl}. This script parses the mol\_N.txt files and produces a table like the one named \begin{verbatim} pixelIntensities.eachFragment.all7134Groups \end{verbatim}
which for a fragment with 4 pixels looks like
\begin{verbatim}
moleculeID      fragmentIndex   fragmentLength  pixelIntensities
2327101_734_2060000     0       3.708   24211   23480   23352   23029
\end{verbatim}
\item
To create the dataset to be read into R, do the following:\\
\begin{verbatim}
./intensityForAlignedFrags.pl -ch 1 -f 2 -a 
alignmentChunks.withLength.all7134Groups.goldOnly 
-nM pixelIntensities.eachFragment.all7134Groups
\end{verbatim}
To be more specific, the following is the complete command, including the correct folder paths
\begin{verbatim}
intensityForAlignedFrags.pl -ch 3 -f 13762 -a 
/aspen/steveg/human_nMaps/GC_content/alignmentChunks.withLength.all7134Groups.goldOnly 
-nM /aspen/steveg/human_nMaps/GC_content/subdivideFragments/
pixelIntensities.eachFragment.all7134Groups > chr3_frag13762_intensities
\end{verbatim}

\end{enumerate}
\end{document}
