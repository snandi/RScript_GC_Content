\documentclass[11pt]{article}
%\usepackage{fancyhdr}
\usepackage[hmargin=3cm,vmargin=3cm]{geometry}
\usepackage{dsfont}
\usepackage{bbold}
\usepackage{graphicx, float}
\usepackage{verbatim}
\usepackage{amssymb, amsmath}
%\usepackage[hmargin=3cm,vmargin=3cm]{geometry}
\usepackage{wrapfig}
\DeclareGraphicsExtensions{.pdf,.png,.jpg, .jpeg}

\usepackage{Sweave}
\begin{document}
\Sconcordance{concordance:VariableDefinitions.tex:VariableDefinitions.Rnw:%
1 12 1 1 0 35 1}


\title{Variable definitions\\ Project: Relationship between Intensity and GC Content}
\author{Subhrangshu Nandi\\
  Department of Statistics\\
  Laboratory of Molecular and Computational Genomics\\
  nandi@stat.wisc.edu}
\date{July 13, 2013}
\maketitle

\subsection*{moleculeID}
{\emph{moleculeID}} uniquely identifies a molecule in the database. It has three parts, example: 23705\_734\_2030001. The first part indicates $GroupID$, the second part indicates $RunID$. So, if it is from run $0$, it will have $734$ in the second part, indicating version 3.4 of INCA software. The third part is the stretch factor of the molecule number. So $2030001$ indicates a stretch factor 203 of molecule 1 of group 23705.

\subsection*{alignedChr}
Aligned Chromosome number. For human it will vary between 1 and 23.

\subsection*{alignedFragIndex}

\subsection*{fractionGC}

\subsection*{intensityPerPixel}

\subsection*{fragConversionFactor}

\subsection*{lengthRatio}

\subsection*{moleculeConversionFactor}

\subsection*{numFrags}

\subsection*{numPixels}



\end{document}
